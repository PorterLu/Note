\documentclass[twocolumn]{ctexart}
\usepackage{listings}
\usepackage{clrscode}
\usepackage{amsmath}
\usepackage{siunitx}
\usepackage{booktabs}
\usepackage{graphicx}
\usepackage{caption}
\usepackage{bicaption}
\usepackage{balance}
\balance
\bibliographystyle{plain}
\title{一起学习 \LaTeX}
\author{porter \\123@qq.com \and university}
\date{\today}
\begin{document}
    \maketitle
    \tableofcontents
    \clearpage
    Hello world!

    Hello again

    \section{Title}
        \subsection{subTitle}
            \subsubsection{subsubtitle}
                \subparagraph{today, we are going to learn about paper typesetting}

            \subsubsection{转义}

        \subsection{列表}
            \subsubsection{item}
                \begin{itemize}
                    \item itemize
                    \item enumerate
                    \item description
                \end{itemize}
            \subsubsection{enumerate}
                \begin{enumerate}
                    \item itemize
                    \item enumerate
                    \item description
                \end{enumerate}
            \subsubsection{description}
                \begin{description}
                    \item[itemize] dot
                    \item[enumerate] num
                    \item[description] des
                \end{description}

            \subsection{code}
                \subsubsection{short code}
                    \verb |#include <stdio.h>|
                \subsubsection{long code}
                    \begin{verbatim}
                        #include <stdio.h>
                        int main(void){
                            printf("hello world\n");
                        }
                    \end{verbatim}
                
                    \begin{lstlisting}[language=C]
                        #include <stdio.h>
                        int main(void){
                            printf("hello world\n");
                        }
                    \end{lstlisting}

                \subsubsection{data structure}
                    \begin{codebox}
                        \Procname{$\proc{Merge-Sort}(A, p, r)$}
                        \li \If $p<r$
                        \li \Then $q \gets \lfloor(p+r)/
                        2\rfloor$
                        \li $\proc{Merge-Sort}(A, p, q)$
                        \li $\proc{Merge-Sort}(A, p+1, q)$
                        \li $\proc{Merge-Sort}(A, p, q, r)$
                            \End
                    \end{codebox}
            \subsection{注脚}
            欧几里德\footnote{数学家}
        \section{数学公式}
            \emph{数学公式必须在数学模式下输入}
            \subsection{数学结构}
                \subsubsection{行内公式}
                    爱因斯坦提出了质能方程: $E = MC^2$
                \subsubsection{显示公式}
                $$x_{1,2} = \frac{-b \pm \sqrt[2]{b^2 - 4ac}}{2a}$$
                \begin{equation}
                    F_1 = F_2 = \frac{Gm_1m_2}{r^2}
                \end{equation}
                \begin{equation}
                    \int_a^b f(x) \mathrm{d}x = F(x) |_a^b = F(b) - F(a)
                \end{equation}
                \[
                    \begin{bmatrix}
                        1 & 0 & \cdots & 9 \\
                        0 & 1 & \cdots & 0 \\
                        \vdots & \vdots & \ddots & \vdots \\
                        0 & 0 & \cdots & 1 \\
                    \end{bmatrix}  
                \]

                \num{-1.23e45}{m/s}
                \SI{2999999999}{m/s}\\
                \begin{tabular}{|S|}
                    \hline 
                    -2345\\
                    \hline
                    123\\
                    \hline
                \end{tabular} 
            \subsection{图表}
                \begin{tabular}{ccc}
                    \hline
                    left & center & right \\
                    \hline
                    文本左对齐 & 居中对齐 & 右对齐 \\
                    \hline
                \end{tabular}
                \begin{table}[h]
                    \centering
                    \caption{标题}
                    \label{表格}
                    \begin{tabular}{cccc}
                        \hline
                        \bfseries Do & \bfseries You  & \bfseries Love & \bfseries Me \\
                        \hline
                        Yesterday & Yes & Yes & Yes \\
                        Today & of Course & of Course & of Course \\
                        Tomorrow & Yes & Yes & Yes \\
                        \hline
                    \end{tabular}
                \end{table}
                \subsubsection{another table}
                another table described below, another table described below, another table described below,
                another table described below, another table described below, another table described below,
                another table described below, another table described below, another table described below

                \begin{table}[h] % here, top, bottom, page
                    \centering
                    \begin{tabular}{cccc}
                        \toprule
                        \bfseries Do & \bfseries You  & \bfseries Love & \bfseries Me \\
                        \midrule
                        Yesterday & Yes & Yes & Yes \\
                        Today & of Course & of Course & of Course \\
                        Tomorrow & Yes & Yes & Yes \\
                        \bottomrule
                    \end{tabular}
                \end{table}
                \begin{figure}
                    \centering
                    \caption{标题}
                    \includegraphics[scale = 0.3]{4-1-1.png}
                \end{figure}

                \bibliography{baiduxueshu_papers_20221001220957.bib}
                \nocite{*}
\end{document}